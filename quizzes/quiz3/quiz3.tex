\documentclass{article} 
\usepackage{listings}
\title{CMPT 202\\Quiz 3}
\begin{document}
\maketitle
\emph{Name:}

\emph{Show your work. Proving $f(n)$ has big O complexity $g(n)$ means that
$Kg(n)$ is always greater than $f(n)$ from some value $n_0$ onwards. $K$ is
usually a big positive number.}

\begin{enumerate}
	\item What is the maximum number of nodes a complete binary can have if
	its height is 2? (A tree of height 2 has three levels.) How many in a
	tree of height 8? (You may leave your answer in the form $X^Y$)
	\vspace{1in}
	\item Prove that $f(n) = n - 1$ has big O complexity $O(n)$.
	\vspace{1in}
	\item Prove that $f(n) = n$ has big O complexity $O(n^2)$
	\vspace{1in}
	\item \emph{Circle the correct answer: }The big O complexity of binary
	search on an array is: \hspace{0.3in} O(1) \hspace{0.3in} O(log $n$) \hspace{0.3in}
	O($n$) \hspace{0.3in} O($n^2$) \hspace{0.3in} other . If you circled
	\emph{other}, write your answer here:
	\item In a binary search tree, the root node contains 5 as data. What
	possible values can the right subtree contain?
	\vspace{1in}
	\item Write an example of a recursive method below. Label the base case
	and the recursive step(s).
	\vspace{1in}
	\item Complete the method below if necessary. The method should perform
	an in-order traversal. \\
	You may assume that the node class has public
	leftChild and rightChild instance variables, and that these instance
	variables are of type Node.
	\begin{lstlisting}[language=Java]
	public static void traverse(Node root) {
		if(root == null) {
			return;
		}
		else {
			

			System.out.println("Visited: " + root.content);


		}
	}
	\end{lstlisting}

	\item Select the data structure that has the fastest lookup time,
	assuming that they all contain the same amount of data :
		\begin{enumerate}
			\item A complete binary search tree
			\item A binary search tree that is not complete
			\item A hash table with frequent collisions
			\item A hash table with no collisions
		\end{enumerate}

\end{enumerate}
\end{document}
