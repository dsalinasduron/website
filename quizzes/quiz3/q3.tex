\documentclass{article} 
\usepackage{listings}
\title{CMPT 202\\Quiz 3}
\begin{document}
\maketitle
\emph{Name:}\\
\begin{enumerate}
	\item Given $n$ nodes, what is the  height of a complete binary tree?
	\item Prove that $f(n) = n + 1$ has big O complexity $O(n)$.
	\item Prove that $f(n) = 100n$ has big O complexity $O(n)$
	\item What is the big O complexity of binary search?
	\item \emph{Fill in the blank:} In a binary search tree, the
	\underline{\hspace{0.6in}}
	child is \underline{\hspace{0.6in}} than the root.
	\item \label{traversal} Is the following tree traversal recursive? If not, explain why.
	If so, label the base case and the recursive step(s)
	\begin{lstlisting}[language=Java]
	public static void traverse(Node root) {
		if(root == null) {
			return;
		}
		else {
			traverse(root.leftChild);
			traverse(root.rightChild);
			System.out.println("Visited: " + root.content);
		}
	}
	\end{lstlisting}
	\item \emph{True or False:} The traversal in question \ref{traversal}
	is a preorder traversal.
	\item Linear probing and separate chaining are two methods to resolve
	collisions in hash tables. Explain how each resolves collisions.
	\item Consider a complete binary search tree and a hash table that is
	almost empty. Also, assume that they contain the same data. Which one
	will give a faster look-up time? Would your answer change depending on
	the shape of the tree or how full the hash table is? Explain how.
\end{enumerate}
\end{document}
