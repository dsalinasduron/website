\documentclass{article}
\usepackage{listings}
\begin{document}
\newcommand{\var}[1]{\texttt{#1}}
\newcommand{\java}[1]{\lstinputlisting[language=java,numbers=left]{#1}}
\section{Java}
	\subsection{Formal \& Actual Type}
		\lstinputlisting[language=java,numbers=left]{q1.java}
	\begin{enumerate}
		\item What is the formal type of variable \var a?
		\item What is the actual type of variable \var b?
		\item Will line 5 compile? Why or Why not? \\ \emph{Hint: Look up the Object
		class in the Java API}
		\item Will line 6 compile? Why or Why not? \\ \emph{Hint: Look
		up the Object class in the Java API}
		\item Assuming that line 6 compiles and that both the Object class
		and the String class have their own implementation of
		\var{toString}, which version runs in line 6? The
		Object implementation or the String implementation?
	\end{enumerate}

	\subsection{Static}
	\java{q2.java}
	\begin{enumerate}
		\item There is an illegal method call in \var main. Which line
		is illegal?
		\item There is an illegal assignment in \var{m1} . Which line is
		illegal?
		\item Which variables can be used in \var{m2}?
	\end{enumerate}

	\subsection{Scope}
	\java{q3.java}
	\begin{enumerate}
		\item This code prints \\ 7 \\ 8 \\ Modify lines 9, 10 and 13 so that the printout is \\ 9 \\ 6
	\end{enumerate}
	\subsection{Generics}
	\java{q4.java}
	\begin{enumerate}
		\item Modify the Node class so that \var{getContent} for \var a
		returns a String and \var{getContent} for \var b returns an
		Integer.
	\end{enumerate}

\section{Data Structures}
	\subsection{Stack}
		\begin{enumerate}
			\item What does it mean to push an item onto the stack?
			\item What does it mean to pop an item from the stack?
			\item If 1,2,3 are pushed onto the stack, in what order
			will they be reversed?
		\end{enumerate}
	\subsection{Queue}
		\begin{enumerate}
			\item What four letter acronym is used to describe the
			order in which items are added to a queue?
			\item What do the four letters mean?
		\end{enumerate}
	\subsection{Heap}
		\begin{enumerate}
			\item In a heap, is the parent of the node greater than
			or less than the node?
			\item Binary trees are used to represent heaps.
			Describe the shape of binary tree a heap is trying to
			maintain.
			\item Can any node be removed from the heap? If not,
			which node(s) can be removed from the heap?
		\end{enumerate}
	\subsection{Dictionary}
		\begin{enumerate}
			\item Dictionaries are a collection of Key-Value pairs.
			The pairs can be stored in the dictionary and then one
			(either the key or the value) can be looked up using
			the other. Are keys retrieved using values or are
			values retrieved using keys?
			\item Dictionaries are usually implemented using a hash
			table for fast lookup times. How does hashing allow for
			a fast lookup time?
		\end{enumerate}

\section{Algorithms}
	\subsection{Sorting}
		\begin{enumerate}
			\item If a list is sorted, what is the first item? What
			is the last item?
			\item Name three sorting algorithms.
			\item The fastest guaranteed sorting algorithm is
			MergeSort. What is its complexity?
		\end{enumerate}
	\subsection{Binary Search}
	\begin{enumerate}
		\item The height of a tree grows as the number of nodes
		increases. If a tree has $n$ nodes, what is its height?
		\item Given your previous answer, what is the complexity of
		looking up a value in a complete binary search tree?
		\emph{Hint: This is the same complexity of searching for an
		item in a sorted array.}
	\end{enumerate}
	\subsection{Recursion}
		\begin{enumerate}
			\item What are the two basic components of all
			recursive functions? 
			\item Lists are recursive data structures. A list can
			be thought of being composed of smaller lists. What is
			another recursive data structure?
		\end{enumerate}
\section{Because for some reason I kept mentioning it:}
\subsection{\var main}
\begin{enumerate}
	\item Why must \var main be static?
\end{enumerate}
\end{document}
