\documentclass{article}
\usepackage{listings}
\begin{document}
\newcommand{\var}[1]{\texttt{#1}}
\newcommand{\java}[1]{\lstinputlisting[language=java,numbers=left]{#1}}
\newcommand{\sol}[1]{\\}
\section{Java}
	\subsection{Formal \& Actual Type}
		\lstinputlisting[language=java,numbers=left]{q1.java}
	\begin{enumerate}
		\item What is the actual type of variable \var a? \sol{Object}
		\item What is the actual type of variable \var b? \sol{String}
		\item Will line 5 compile? Why or Why not? \\ \emph{Hint: Look up the Object
		class in the Java API} \sol{Yes. The Object class has a
		hashCode method.}
		\item Will line 6 compile? Why or Why not? \\ \emph{Hint: Look
		up the Object class in the Java API} \sol{Yes. The Object class
		has a toString method.}
		\item Assuming that line 6 compiles and that both the Object class
		and the String class have their own implementation of
		\var{toString}, which version runs in line 6? The
		Object implementation or the String implementation? \sol{The
		String implementation. The actual type is what decides which
		implementation runs.}
	\end{enumerate}

	\subsection{Static}
	\java{q2.java}
	\begin{enumerate}
		\item There is an illegal method call in \var main. Which line
		is illegal? \sol{Line 12. Static methods can only directly call
		other static methods.}
		\item There is an illegal assignment in \var{m1} . Which line is
		illegal? \sol{Line 5. Static methods can only use static
		variables.}
		\item Which variables can be used in \var{m2}? \sol{x and y.
		Non-static methods can use both static and non-static
		variables.}
	\end{enumerate}

	\subsection{Scope}
	\java{q3.java}
	\begin{enumerate}
		\item This code prints \\ 7 \\ 8 \\ Modify lines 9, 10 and 13
		so that the printout is \\ 9 \\ 6 \\ \sol{Make L.9 print out
		the value of the x that L.13 changes. Change L.13 to this.x = 9; and L.9 to
		System.out.println(instance.x);. To reference a static variable, the class name can
		be used. Change L.10 to System.out.println(q3.y);  }
	\end{enumerate}
	\subsection{Generics}
	\java{q4.java}
	\begin{enumerate}
		\item Modify the Node class so that \var{getContent} for \var a
		returns a String and \var{getContent} for \var b returns an
		Integer. \\ \sol{ Node<T> \\ T content; \\  public Node(T input)
		\{ \\ public T getContent() \{  \\ Node<String> a = new
		Node<String>("Ada"); \\ Node<Integer> b = new Node<Integer>(5); }
	\end{enumerate}

\section{Data Structures}
	\subsection{Stack}
		\begin{enumerate}
			\item What does it mean to push an item onto the stack?
			\sol{The item is the new top of the stack.}
			\item What does it mean to pop an item from the stack?
			\sol{The item is removed from being the top of the
			stack. The previous top becomes the new top.}
			\item If 1,2,3 are pushed onto the stack, in what order
			will they be returned? \sol{In reverse order: 3,2,1}
		\end{enumerate}
	\subsection{Queue}
		\begin{enumerate}
			\item What four letter acronym is used to describe the
			order in which items are added to a queue? \sol{FIFO}
			\item What do the four letters mean? \sol{First in
			First Out}
		\end{enumerate}
	\subsection{Heap}
		\begin{enumerate}
			\item In a max-heap, is the parent of the node greater than
			or less than the node? \sol{The parent is greater than
			the children.}
			\item Binary trees are used to represent heaps.
			Describe the shape of binary tree a heap is trying to
			maintain. \sol{Flat and wide: a complete binary tree.}
			\item Can any node be removed from the heap? If not,
			which node(s) can be removed from the heap? \sol{The
			only node that can be removed is the node that, when
			removed, will maintain the completeness of the tree.
			This is the last node that was added. In terms of
			content, only the content of the root of the heap can
			be removed.  }
		\end{enumerate}
	\subsection{Dictionary}
		\begin{enumerate}
			\item Dictionaries are a collection of Key-Value pairs.
			The pairs can be stored in the dictionary and then one
			(either the key or the value) can be looked up using
			the other. Are keys retrieved using values or are
			values retrieved using keys? \sol{Values are retrieved
			using keys.}
			\item Dictionaries are usually implemented using a hash
			table for fast lookup times. How does hashing allow for
			a fast lookup time? \sol{Hashing allows us to transform
			any key into an index of the array we are using for
			storage. When the get(key) method is called, the key is
			hashed and and the index $i$ of the key is found. Then,
			the value at $i$ is returned. Since hashing is $O(1)$
			and array lookups are $O(1)$, the get method is $O(1)$
			unless there are collisions. }
		\end{enumerate}

\section{Algorithms}
	\subsection{Sorting}
		\begin{enumerate}
			\item If a list is sorted, what is the first item? What
			is the last item? \sol{The first item is the smallest
			item. The last item is the largest item.}
			\item Name three sorting algorithms. \sol{Insertion
			sort, Quicksort, Selection Sort}
			\item The fastest guaranteed sorting algorithm is
			MergeSort. What is its complexity? \sol{$O(n \log n)$}
		\end{enumerate}
	\subsection{Binary Search}
	\begin{enumerate}
		\item The height of a tree grows as the number of nodes
		increases. If a tree has $n$ nodes, what is its height?
		\sol{$\log_2 n$, rounded down.}
		\item Given your previous answer, what is the complexity of
		looking up a value in a complete binary search tree?
		\emph{Hint: This is the same complexity of searching for an
		item in a sorted array.} \sol{The complexity of searching in a
		BST is the same as the number of comparisons that have to be
		done. This is at most the height of the tree, which is itself
		at most $O(log_2 n)$. }
	\end{enumerate}
	\subsection{Recursion}
		\begin{enumerate}
			\item What are the two basic components of all
			recursive functions?  \sol{The base case and recursive
			step.}
			\item Lists are recursive data structures. A list can
			be thought of being composed of smaller lists. What is
			another recursive data structure? \sol{A binary tree.
			It has a root and two smaller trees attached.}
		\end{enumerate}
\section{Because for some reason I kept mentioning it:}
\subsection{\var main}
\begin{enumerate}
	\item Why must \var main be static? \sol{The main method must be run
	straight from the class because, at the beginning of any Java program,
	no objects have been created yet.}
\end{enumerate}
\end{document}
